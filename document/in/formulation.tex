Moving toward the ultimate goal of finding an optimal solution for the problem
stated before, the first step is to formulate this problem is a way solvable by
integer programming. Subsequently these formulae can be easily structured and
implemented such that available integer programming solvers can be utilized to
find our required answer. Here as suggested by course instructors, CPLEX solver
is used to find the proper answer to a mixed integer linear program created and
formulated out of requirements of mentioned problem.

Short after formulating the problem, the implementation is given based on three
different flow control methods, single commodity flows, multi commodity flows
and Miller-Tucker-Zemlin subtour elimination constraints which are used to
eliminate subtours from our solutions.

Starting with the standard TSP formulation, we have to adjust it a little to 
account for the extra properties in TCBVRP. Our basic variables are not
indexed $x_{ij}$ as in two dimensions, but $x_{ijk}$ where $k$ is the number
of the tour. So we need $n*n*m$ variables instead of just $n*n$, and $x_{ijk} \in \{0,1\}$
indicates whether the arch from $i$ to $j$ was taken in tour $k$ (then $x_{ijk} = 1$) or
not ($x_{ijk} = 0$).

After declaring the variables, the next thing which is mandatory is an objective function 
that can be optimized. Here it is proposed to try to minimize following function as our
objective function:

\begin{equation}
\mbox{minimize }\sum_{k = 1}^{m} \sum_{i = 0}^{n} \sum_{j = 0}^{n} c_{ij} x_{ijk}
\end{equation}

Where $c_{ij}$ designates the cost of the connection $ij$.

Now since the objective function is declared, so the constraint must be defined
as well. Let's start with the requirements listed before and formulate them one
by one.

The basic constraints are the same for all three formulations. $D$ denotes the
set of demand nodes, $S$ the set of supply nodes. If not otherwise designated,
then $i, j \in N$ and $k \in M$. We designate the depot as node $0$.

\begin{description}
  \item[Each demand node is visited exactly once.]
    \begin{equation}
      \forall j \in D. \sum_{i,i \neq j, k} x_{ijk} = 1
    \end{equation}
  \item[Each supply node is visited at most once.]
    \begin{equation}
      \forall j \in S. \sum_{i,i \neq j, k} x_{ijk} \leq 1
    \end{equation}
  \item[Each node is left as often as it is visited.]
    \begin{equation}
      \forall j \in N. \forall k \in M. \sum_{i,i \neq j} x_{ijk} = \sum_{l \in N, l \neq j} x_{jlk}
    \end{equation}
    This formulation also ensures that a certain node is visited
    and left in the same tour. If we had just mimicked the two 
    preceding constraints for outgoing connections as well,
    then it would be possible to visit a node in one tour
    and leave it in the other.        
  \item[The depot is visited once in each tour.]
    \begin{equation}
      \forall k. \sum_{i} x_{i,0,k} = 1
    \end{equation}
  \item[Each connection is either taken or not.]
    \begin{equation}
        \forall i,j \in N. \forall k \in M. \begin{cases} 
          &x_{ijk} \in \{0,1\} \mbox{ iff. $ij$ allowed} \\
          &x_{ijk} = 0 \mbox{ otherwise}
      \end{cases}
    \end{equation}
    Allowed connections are those allowed in TCBVRP, e.g.
    \begin{itemize}
      \item $Supply \rightarrow Demand$
      \item $Demand \rightarrow Supply$
      \item $Demand \rightarrow Depot$
      \item $Depot \rightarrow Supply$
      \item $Depot \rightarrow Depot$
    \end{itemize}

    Note that $Depot \rightarrow Depot$ is allowed, thus making empty tours possible. 
    This means that there is intentionally no $i \neq j$  property in the above 
    formula while it is present in the formulation
    of the original TSP problem. All other self-connections are forbidden by the 
    $allowed$ rule above though.

  \item[No subtour can be longer than $T$.]
    \begin{equation}
      \forall k. \sum_{i} \sum_{j} x_{ijk} c_{ij} \leq T
    \end{equation}
\end{description}
