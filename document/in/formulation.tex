Moving toward the ultimate goal of finding an optimal solution for the problem
stated before, the first step is to formulate this problem is a way solvable by
integer programming \cite{garfinkel1972integer}. Subsequently these formulae can be easily structured
and implemented such that available integer programming solvers can be utilized to
find our required answer. Here as suggested by course instructors, IBM ILOG
CPLEX Solver \cite{cplex2009v12}
is used to find the proper answer to a mixed integer linear program created and
formulated out of requirements of mentioned problem.

Short after formulating the problem, the implementation is given based on three
different flow control methods, single commodity flows, multi commodity flows
and Miller-Tucker-Zemlin subtour elimination constraints which are used to
eliminate subtours from our solutions.

Starting with the standard TSP \cite{miller1960integer} formulation, we have to adjust it a
little to account for the extra properties in TCBVRP. Our basic variables are not
indexed $x_{ij}$ as in two dimensions, but $x_{ijk}$ where $k$ is the number
of the tour. So we need $n*n*m$ variables instead of just $n*n$, and $x_{ijk} \in \{0,1\}$
indicates whether the arch from $i$ to $j$ was taken in tour $k$ (then $x_{ijk} = 1$) or
not ($x_{ijk} = 0$).

After declaring the variables, the next thing which is mandatory is an objective function 
that can be optimized. Here it is proposed to try to minimize following function as our
objective function:

\begin{equation}
\mbox{minimize }\sum_{k = 1}^{m} \sum_{i = 0}^{n} \sum_{j = 0}^{n} c_{ij} x_{ijk}
\end{equation}

Where $c_{ij}$ designates the cost of the connection $ij$.

Now since the objective function is declared, so the constraint must be defined
as well. Let's start with the requirements listed before and formulate them one
by one.

The basic constraints are the same for all three formulations. $D$ denotes the
set of demand nodes, $S$ the set of supply nodes. If not otherwise designated,
then $i, j \in N$ and $k \in M$. We designate the depot as node $0$.

\begin{description}
  \item[Each demand node is visited exactly once.]
    \begin{equation}
      \forall j \in D. \sum_{i,i \neq j, k} x_{ijk} = 1
    \end{equation}
  \item[Each supply node is visited at most once.]
    \begin{equation}
      \forall j \in S. \sum_{i,i \neq j, k} x_{ijk} \leq 1
    \end{equation}
  \item[Each node is left as often as it is visited.]
    \begin{equation}
      \forall j \in N. \forall k \in M. \sum_{i,i \neq j} x_{ijk} = \sum_{l \in N, l \neq j} x_{jlk}
    \end{equation}
    This formulation also ensures that a certain node is visited
    and left in the same tour. If we had just mimicked the two 
    preceding constraints for outgoing connections as well,
    then it would be possible to visit a node in one tour
    and leave it in the other.        
  \item[The depot is visited once in each tour.]
    \begin{equation}
      \forall k. \sum_{i} x_{i,0,k} = 1
    \end{equation}
  \item[Each connection is either taken or not.]
    \begin{equation}
        \forall i,j \in N. \forall k \in M. \begin{cases} 
          &x_{ijk} \in \{0,1\} \mbox{ iff. $ij$ allowed} \\
          &x_{ijk} = 0 \mbox{ otherwise}
      \end{cases}
    \end{equation}
    Allowed connections are those allowed in TCBVRP, i.e.
    \begin{itemize}
      \item $Supply \rightarrow Demand$
      \item $Demand \rightarrow Supply$
      \item $Demand \rightarrow Depot$
      \item $Depot \rightarrow Supply$
      \item $Depot \rightarrow Depot$
    \end{itemize}

    Note that $Depot \rightarrow Depot$ is allowed, thus making empty tours possible. 
    This means that there is intentionally no $i \neq j$  property in the above 
    formula while it is present in the formulation
    of the original TSP problem. All other self-connections are forbidden by the 
    $allowed$ rule above though.

  \item[No subtour can be longer than $T$.]
    \begin{equation}
      \forall k. \sum_{i} \sum_{j} x_{ijk} c_{ij} \leq T
    \end{equation}
\end{description}

\subsection{Single Commodity Flow (SCF)}

We need the additional constraints in SCF, MCF and MTZ to avoid subtours. The idea n SCF is
to send out some commodity from the depot and have each reached node subtract one. Thus, a 
subtour which does not include the depot is not possible because no flow would be coming in
and therefore the nodes inside the subtour would have negative flow, which is forbidden.

We introduce a new set of variables $f_{ijk}$, corresponding to $x_{ijk}$ describing the units
of flow over a certain arch $ij$ in a tour $k$. If $x_{ijk} = 0$, then $f_{ijk} = 0$, too.

\begin{description}
  \item[The depot sends out $|D|$ units of flow.] $D$ here is the set of demand nodes. We do 
  not need to send out flow for supply nodes because they do not necessarily have to be
  reached at all, and they cannot form subtours on their own because it's not allowed to 
  go from a supply node to anywhere but a demand node.
     \begin{equation}
       \sum_{j>0,k} f_{0jk} = D
     \end{equation}

  \item[A node is either reached in a certain tour or not.]
     Compared to the standard TSP problem we have to deal with a complication here: A node which 
     is not reached in a certain tour does not get any flow and therefore cannot subtract one,
     so we have to distinguish between reached nodes and unreached nodes, giving rise to a new
     variable $r_{jk} \in \{0,1\}$ and the following constraint.
     \begin{equation}
        \forall j>0. \forall i \neq j. \forall k. r_{jk} \geq x_{ijk}
     \end{equation}
     This ensures that $r_{jk}$ is $1$ if $j$ is reached, but it cannot also be $1$ if it unreached,
     because this would mean that $j$ would be forced to subtract a flow unit which it cannot
     do because there would be no incoming flow.

  \item[A demand node subtracts a unit of flow if it reached in a tour, and passes it on unmodified
        otherwise.] We can now formulate this crucial property.
  \begin{equation}
        \forall j>0 \in D. \forall k. \sum_{i \neq j} f_{ijk} = \sum_{l \neq j \in N} f_{jlk} = r_{jk}
  \end{equation}

  \item[Any other node passes on flow unmodified.]
  \begin{equation}
        \forall j>0, j \in N \setminus D. \forall k. \sum_{i \neq j} f_{ijk} = \sum_{l \neq j \in N} f_{jlk} = 0
  \end{equation}

  \item[flow is at most $|D|$ or $0$] 

  If an arch is not taken, then the flow along it must be zero. Otherwise it may be positive, but 
  not more than $|D|$.

  \begin{equation}
        \forall i. \forall j \neq i. \forall k. 0 \leq f_{ijk} \leq |D| x_{ijk}
  \end{equation}
\end{description}

\subsection{Multi Commodity Flow (MCF)}

This is very similar to SCF, but we need $|D|$ times more variables, because we send out $|D|$ commodities, and only one unit of each. Thus, our $f_{ijk}$ variables turn into $f_{ijkc}$, where $c$ is the index of a demand node corresponding to a commodity.

\begin{description}
  \item[The depot sends out $1$ unit of flow per demand node.]
     \begin{equation}
       \forall c \in D. \sum_{j>0,k} f_{0jkc} = 1
     \end{equation}

  \item[There is no flow going back to the depot.]
     \begin{equation}
       \forall c \in D. \sum_{i>0,k} f_{i0kc} = 0
     \end{equation}

  \item[A node is either reached in a certain tour or not.]
     \begin{equation}
        \forall j>0. \forall i \neq j. \forall k. r_{jk} \geq x_{ijk}
     \end{equation}

  \item[A demand node subtracts a unit of flow if it reached in a tour, and passes it on unmodified
        otherwise.] We can now formulate this crucial property.
  \begin{equation}
        \forall j>0 \in D. \forall k. \sum_{i \neq j} f_{ijk} = \sum_{l \neq j \in N} f_{jlk} = r_{jk}
  \end{equation}

  \item[Any other node passes on flow unmodified.]
  \begin{equation}
        \forall j>0, j \in N \setminus D. \forall k. \sum_{i \neq j} f_{ijk} = \sum_{l \neq j \in N} f_{jlk} = 0
  \end{equation}

  \item[flow is at most $|D|$ or $0$] 

  If an arch is not taken, then the flow along it must be zero. Otherwise it may be positive, but 
  not more than $|D|$.

  \begin{equation}
        \forall i. \forall j \neq i. \forall k. 0 \leq f_{ijk} \leq |D| x_{ijk}
  \end{equation}


\end{description}
