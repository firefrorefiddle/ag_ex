In this section, each part is described in terms of implementation. All codes
are in {\it C++} language using CPLEX Solver.
The first important part would be the objective function.

As it is shown in the code above, there is a decision variable called
\texttt{ max } used to eliminate all unwilling cases from possible selected
possibilities. For instance this variable is \texttt{ 1 } for a connection from
a node $D$ to a node $S$ or vice versa but \texttt{ 0 } for a connection between
two nodes with the same type since we don't want to have any direct connection
from a $S$ node to another $S$ node or any $D$ node to another $D$ node.

Moving, we get to the constraints. In order to keep this report short enough,
only one samples of how the constraints are implemented, is presented and the
rest lie in the code which is enclosed and fully documented and can be found at
its GitHub repository\footnote{\url{https://github.com/firefrorefiddle/ag_ex}}.
As an instance, we know that sum of incoming connections for each node should
be exactly equal to the sum of outgoing connections from that node. This
constraint is implemented as shown in Appendix\ref{app:app01}.

The other important part of the implementation is eliminating subtours from
results. Such subtours are not connected to the node \texttt{depot}. Thus it is
decided to emit the initial flow from the node \texttt{depot} to all tours
simultaneously and evaluate the inflow of the  \texttt{depot} at the end. On the
other hand, by characteristics of the problem we know that only one of the nodes
$D$ or $S$ alone cannot appear in a path and they come consecutively. Thus in
the implementation only half of the nodes in a tour which are $D$ nodes are
considered as consumers of flow and therefore $\frac{l_k}{2}-1$, $l_k$ denoting
the number of participant nodes in each tour, is the flow amount emitted from
the \texttt{depot} for that tour.
Finally:

\begin{center}
$\sum_{k=0}^{m}{\frac{l_k}{2}-1} = |D| - 1$
\end{center}

\subsection{Single commodity flows}

So for {\it single commodity flows} we have counted all connected $D$ nodes to
the \texttt{depot} over all tours and it must be exactly $|D|$ nodes in whole
system.

A simplified version of the implementation of this part is demonstrated in
Appendix\ref{app:app02}.

\subsection{Multi commodity flows}


Multi commodity flows is considerable similar to single commodity flows. The
only difference is that this time, \texttt{ depot } sends out only $p$ flows out
but $n$ times, in our case $\#D$ times, and expects $p-1$ back. Consequently,
wrapping the code used for single commodity flows in a loop which iterates $\#D$
times and deciding over \texttt{(sumDepotOut - sumDepotIn) == 1} results in
elimination of subtours using multi commodity flows.

\subsection{Miller-Tucker-Zemlin subtour elimination constraints}

