In this section, each part is described in terms of implementation. The first
important part would be the objective function.

\begin{lstlisting}
for (u_int i=0; i<n; i++) //All nodes in tale
{
    for (u_int j=0; j<n; j++) //All nodes in head
    {
    	for (u_int k=0; k<m; k++) //All tours
        {
            //Choice of using connection between i and j
            x[i, j, k] = IloIntVar(env, 0, max, xname.str().c_str());
            
            //Sum over the cost of all connections
            totalCosts += (x[i,j,k] * Distance(i,j));
        }
    }
}
...
model.add(x);
model.add(IloMinimize(env, totalCosts));
     
\end{lstlisting}

As it is shown in the code above, there is a decision variable called
\texttt{ max } used to eliminate all unwilling cases from possible selected
possibilities. For instance this variable is \texttt{ 1 } for a connection from
a node $D$ to a node $S$ or vice versa but \texttt{ 0 } for a connection between
two nodes with the same type since we don't want to have any direct connection
from a $S$ node to another $S$ node or any $D$ node to another $D$ node.

Moving, we get to the constraints. In order to keep this report short enough,
only one samples of how the constraints are implemented, is presented. As an instance, we know that sum of incoming connections for
each node should be exactly equal to the sum of outgoing connections from that
node. This constraint is implemented as following:

\begin{lstlisting}
for(u_int j=0; j<n; ++j) //all nodes
{
    for(u_int k=0; k<m; ++k) //in all tours 
    {
        IloExpr incomingConnections(env);
        for(u_int i=0; i<n; ++i) //sum of incomming connections
        {
            incomingConnections += x[i,j,k];
        }
        IloExpr outgoingConnections(env);
        for(u_int l=0; l<n; ++l) //sum of outgoing connections
        {
            outgoingConnections += x[j,l,k];
        }
        model.add(incomingConnections == outgoingConnections); //equal
    }
}
\end{lstlisting}
